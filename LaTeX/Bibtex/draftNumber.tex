\documentclass[11pt,a4paper]{article}
\usepackage[a4paper,total={6.00in, 8.75in}]{geometry}
\usepackage{graphicx}
\usepackage{natbib}
\usepackage{amsmath,amssymb,xcolor}
\usepackage[breaklinks]{hyperref}  %%%% for line break in \cite{..}

\linespread{1.5}

\usepackage{amsthm}
\newtheorem{theorem}{Theorem}
\newtheorem{lemma}[theorem]{Lemma}

%===============================================
\title{\textbf{A note on the $g$ and $h$ control charts}}
\author{\textsf{Chanseok Park}\\
Applied Statistics Laboratory\\
Department of Industrial Engineering\\ Pusan National University\\
Busan 46241, Korea
\and
\textsf{FirstName LastName}\\
Department of Statistics \\
University of Texas \\
San Antonio, TX 78249, USA
}
\date{}

%%===============================================
\begin{document}
%%===============================================
\maketitle
%------------------------------------------------
\begin{abstract}
%------------------------------------------------
In this note,  ................................

\noindent {\it Keywords:} control charts, geometric distribution,
maximum likelihood estimator, minimum variance unbiased estimator, $g$ and $h$ charts.

%------------------------------------------------
\end{abstract}
%------------------------------------------------



%======================================
\section{Introduction}
%======================================
Introduction here....



%======================================
\section{Basic ......} \label{SEC:basic}
%======================================
For example, Beyyeyan \cite{Benneyan:1999,Benneyan:2000}, to name just a few ......

It is iemediate from \cite{Lehmann/Casella:1998} that .... 

We can employ the Rao-Blackwell theorem \citep{Rao:1945,Blackwell:1947} ......

For example, Minitab \cite{MinitabSupport20} uses .......

It should be noted that the R language provides the \texttt{hypergeo} package to calculate the hypergeometric function; 
see \cite{Hankin:2016}.

In his thesis~\citep{Gao:2020a}

\cite{Jeong/Son/Lee/Kim:2018}

\cite{Ouyang/Park/etc:2019}

\cite{Eun/etc:1989}


%======================================
\section{Equation......} \label{SEC:Equation}
%======================================

In Section~\ref{SEC:basic}, we reviewed .... In this section, we will solve some equations.

We have 
\begin{equation}
x + y = 0
\label{EQ:xyzero}
\end{equation}
Substituting $x=-3$ into (\ref{EQ:xyzero}), we have 
\[
y=3.
\]

However, if one use $x=-10$ in (\ref{EQ:xyzero}), then we have 
\[
y=10.
\]


We solved an equation. In Section~\ref{SEC:Application}, we will incorporate this result into several engineering applications.

%======================================
\section{Application ......} \label{SEC:Application}
%======================================




%%===============================================
% \bibliographystyle{chicago}
  \bibliographystyle{unsrt}
\bibliography{REF.bib}
%%===============================================



%%===============================================
\end{document}
%%===============================================

