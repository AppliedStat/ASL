%----------------------------------------------------
\subsection{Weibull random variable}
%----------------------------------------------------

\begin{frame}
\frametitle{Weibull random variable}
\begin{itemize}
\item One of the most popular distributions used to model the lifetimes and reliability data
is the Weibull distribution, named after a Swedish \textbf{mechanical engineer} by the name of
Walodie Weibull~\citeyear{Weibull:1939}.
\item Indeed, this distribution is as central to the parametric analysis of 
\textbf{reliability engineering} and \textbf{survival} data as the normal distribution in statistics.
\end{itemize}
\end{frame}

%------
\begin{frame}
\frametitle{Weibull random variable}
\begin{block}{Definition}
A random variable $X$ is called \textsf{Weibull} with shape $\alpha$
and scale $\beta$
if its cumulative distribution function is given by
$$
F(x) = 1 - \exp\Big\{ - \big( \frac{x}{\beta} \big)^{\alpha}   \Big\},
\qquad x \ge 0.
$$
\end{block}
It is easily shown that its pdf is given by
$$
f(x) =  \frac{\alpha x^{\alpha-1}}{\beta^\alpha} 
        \exp\Big\{ - \big( \frac{x}{\beta} \big)^{\alpha} \Big\}.
$$
The mean of the Weibull random variable is
$\beta\cdot\Gamma\big[ (\alpha+1)/\alpha \big]$, where
$\Gamma(\alpha)=\int_0^{\infty} x^{\alpha-1} e^{-x} dx$.
\end{frame}
%------

%------
\begin{frame}
\frametitle{Weibull random variable}
\begin{block}{Reparameterized Form}
The following reparameterized form is often used
%---------
\begin{align*}
F(x) = 1 - \exp\big( - \lambda x^{\alpha}  \big)  \textrm{~~and~~} 
f(x) = \lambda\alpha x^{\alpha-1} \exp\big( - \lambda x^{\alpha}  \big),
\end{align*}
\end{block}
%---------
where $\lambda=\beta^{-\alpha}$. 
\begin{itemize}
\item The parameter $\lambda$ is called the \textsf{rate parameter}\index{rate parameter}.
\item It should be noted that with the shape parameter $\alpha=1$, the Weibull distribution
becomes the exponential distribution\index{exponential distribution}
with mean $\beta$ (or rate $\lambda$).
\end{itemize}
\end{frame}
%------


%----------------------------------------------------
\subsection{How to draw Weibull probability plot}
%----------------------------------------------------

%------
\begin{frame}
\frametitle{How to draw Weibull probability plot}

{\Large\bf Basic Idea:} Linearizing the CDF.

\end{frame}
%------

%------
\begin{frame}
\frametitle{How to draw Weibull probability plot}
The Weibull cdf is given by
$$
F(x) = 1 - \exp\big( - \lambda x^{\alpha}  \big). 
$$
Then we have
$$
\log(1-p) = -\lambda x_p^{\alpha}, 
$$
where $p=F(x)$. It follows that
$$
\log\big\{ -\log(1-p) \big\} = \log\lambda + \alpha \log x_p.
$$
This implies that the plot of  
\begin{block}{}
\begin{center}
\fbox{$\log\big\{ -\log(1-p) \big\}$} versus \fbox{$\log x_p$} 
\end{center}
\end{block}
draws a straight line with the slope $\alpha$ and the intercept $\log\lambda$.
The widely-used Weibull \textit{probability paper}\index{probability paper}
is based on this idea.
\end{frame}
%------



\begin{frame}
\frametitle{How to draw Weibull probability plot}
\begin{itemize}
\item Need to find $p=F(x)$ and $x_p$ with real experimental data. 
That is, we need to estimate  $p=F(x)$ and $x_p$ in the following plot:
\item The empirical cdf $\hat{F}(x)$ is used for $p=F(x)$ which is an increasing
 step function jumping $1/n$ at
$x_{(1)}, x_{(2)}, \ldots, x_{(n)}$, where $x_{(i)}$ is sorted from the smallest. 
\item Thus $\hat{p}_i=\hat{F}(x_{(i)})$ has values, $1/n, 2/n, \ldots, n/n$. 

\begin{block}{}
\begin{center}
\fbox{$\log\big\{ -\log(1- \hat{p}_i ) \big\}$} versus \fbox{$\log x_{(i)} $}.
\end{center}
\end{block}
 Actually, \cite{Blom:1958} method is more popular, which uses 
$\hat{p}_i= (i-0.375)/(n+0.25)$ to have better power -- plotting position problem.
\end{itemize}
\end{frame}
%------

